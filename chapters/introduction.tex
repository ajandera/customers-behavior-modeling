% !TEX root = ../thesis.tex

\chaptermark{Introduction}
\phantomsection
\addcontentsline{toc}{chapter}{Introduction}

\chapter*{Introduction}
The goal of this bachelor thesis is to create mathematical model to predict customer’s behavior with focus on e-commerce.
Anonymized data from Megaplay s.r.o stores, both online and offline stores, open data from google.com,
heureka.sk, heureka.cz and statistical data offered by government from years 2018 to 2019 will serve as the initial data
and then we will predict the income for year 2020.
As the baseline of the prediction we will calculate linear and polynomial models fitting and prediction.
The reference data are described in section~\ref{sec:poly}.
Then we will use behavior models which will be composed especially to simulate customer behavior which is responsible for
buying action with specific point which is important in e-commerce business.
As a model demonstration we will use  Live script created in Matlab \footnote{MATLAB is a fourth-generation programming
language and numerical analysis environment.
Uses for MATLAB include matrix calculations, developing and running algorithms, creating user interfaces (UI) and data visualization.}
which will have preprocessed income data at the input and all constants needed to create prediction of future income as output with graphical demonstration of prediction.
As a reference method to our model results from linear and polynomial model will be used.
Then we compare our model results with our set up baselines.
For each method we will calculate aberration of the prediction by predefined criteria.

\section*{Task formulation}

Proposed a mathematical model for prediction of customer behavior consist of vendor, psychology and loyalty sub-models combined
in hidden markov model to final prediction mechanism which should have better performance than Linear and Polynomial regression models.
