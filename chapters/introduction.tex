% !TEX root = ../thesis.tex

\chaptermark{Introduction}
\phantomsection
\addcontentsline{toc}{chapter}{Introduction}

\chapter*{Introduction}
The goal of my bachelor thesis is to create Mathematical model to predict customer’s behavior with focus mainly on e-commerce.
As initial data we will use anonymize data from Megaplay s.r.o stores, both online and offline stores, and open data from google.com,
heureka.sk, heureka.cz and statistical data served by government from years 2018 to 2019 and then we will try to predict the income for year 2020.
As the baseline of the prediction we will calculate linear and polynomial models fitting and prediction.
Explanation of reference data see in section~\ref{sec:poly}.
Then we will use behavior models which will be composed especially to simulate customer behavior which is responsible for
buying action with specific point which are important in e-commerce business.
As a model demonstration we will use  Live script created in Matlab \footnote{MATLAB is a fourth-generation programming
language and numerical analysis environment.
Uses for MATLAB include matrix calculations, developing and running algorithms, creating user interfaces (UI) and data visualization.}
which will have preprocessed income data and all constants needed to create prediction of future income as output with graphical demonstration of prediction.
As a reference method to our model will be used results from linear and polynomial model.
Then we compare our model results with our set up baselines.
For each method we will calculate aberration of the prediction by directly defined criteria.

\section*{Task formulation}

Creating a mathematical model to predict customer behavior consist of vendor, psychology and loyalty sub-models combine
in hidden markov model to final prediction mechanism which should have better results than Linear and Polynomial regression models.
