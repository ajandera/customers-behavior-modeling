% !TEX root = ../thesis.tex

\chapter{Summary} \label{summary}
In our work we focused on creating mathematical models to predict customer behavior based on previous store data and open data available in online tools.
As a baseline we used linear and polynomial regression which is generally used to predict this kind of data.
At the first sight it looks like that the better reference is polynomial model prediction as you can see in the table with results~\ref{compareresults},
but the trend prediction from the simpliest linear regression is useful for some approaches.
Because of easier to solve the model.
In case that we have only this two models as a primary goal, it will be good idea to test their cooperation and use one to correct another.
It should generate a very interesting results with trends prediction and income with small aberration too.
Polynomial prediction has in some months worst result than we expected, this was caused by pandemic situation in the year 2020.
This situation was absolutely unpredictable.
However our set goal was to create a better mechanism to predict this important data for each business based on e-commerce solutions.
To pass our goal we used mechanisms for behavior prediction and mathematical approaches which are usually used in other industries like an autonomous driving,
telecommunications or idea which is looking for a correlation between introduce a new products on market with global consumer behavior.
Mechanisms and approaches that we have learned during the work on this thesis give us the ability to create our own vendor,
psychology and loyalty models which are the base for the final prediction model and help us to finish the goal which we have sat at the beginning of the work.
Our main goal was passed by successful prediction with aberration from real income about 6,6\% for the year 2020 and R-square 0,95.
In order to have best results from our model against the other one we would need more time for preparing data to prediction model.
If the polynomial model could use more data for training, it would return better results, but it was not the goal of our work.
Our goal was fulfilled in cooperation with the Megaplay s.r.o which showed us its business and industry and helped set constants for the prediction model with a minimal deflection.
Therefore the next steps should be clearly define mechanisms to set independent industry constants only on ordinary accessible data without consultation with business owners.
Then it will be good to minimize random generated number with a matrix create with real data of store.
We estabilished competitors matrix with real data to compare real store and vendor power and customer satisfaction.
Unfortunately it opens lot of new problems which have to be solved like a vendor strength comparation across industries.
The situation in the industry was change in an unexpected way and modeling such a situation was beyond this thesis.
The results of the thesis are good enough to show that standard simple statistics approaches are not enough for prediction
customer behavior in a 21st century, so we have to look at the sophisticated approaches like a mathematical modeling.
The evolution of computer technology gives the power to get more complicated calculation to more users and it should be used
to get the more relevant data for entrepreneurs to improve their businesses.
Our prediction is actually combination of our models inside statistical solution like Hidden Markov model.
In the future it will be the right way to leave all statics mechanism and as a combination approach use differential equation
which should be able to serve prediction vector to get results lilke a HMM.
However, when this equation will be found, it opens up a new way of predicting store data from predefined states to dynamically
updated states, and displays real-time results to reflect real visitors and revenue in each store.
From different simulation using our script and update constants we get presumed results but not surprising results for us
The number of predicted orders is dependent on numbers of store visitors.
Customer satisfaction with the combination with price index is the most important value for the prediction and it is corespondent
with real behavior, but this is not generally truth.
It can be aplied to our Central Europe region.
However in the future will be better to have data from different countries or continents to check our model in more global way.
We should expect that the weights ratio will be different in other markets, it will depend on psychology of specific customer and behavior in different countries
