% !TEX root = ../thesis.tex '

\chapter{Modeling different situation} \label{apendixc}

In this section we can see modeling in a different situation.

\section{Increase visitors for 10\%}
When we increase visitors for 10\% to 2592 visitors per day, it will increase our income to 2 687 593 Kč so it's incrase income about 10.47\%.
From this simulation we can see that increasing the number of visitors have positive influence to the store strength.\\
\section{Increase store strength for 5\%}
When we increase store strength for 5\%, it will increase our income to 2 447 891 Kč so it's increase income about 0.62\%.
From this simulation we can see that increasing the store strength (when the previous strength is more than 90\%) has minimal influence for our income.\\
\section{Increase Customer satisfaction for 5\%}
When we increase store Customer satisfaction for 5\%, it will increase our income to 2 435 548 Kč so it's increase income about 0.1\%.
From this calculation we can see that this increase we have to not take into account.
This prediction confirms the idea that everything more than 90\% it has minimal influence for customer decision.
