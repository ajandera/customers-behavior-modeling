%% -----------------------------------------------------------------
%% This file uses UTF-8 encoding
%%
%% For compilation use following command:
%% latexmk -pdf -pvc -bibtex thesis
%%
%% -----------------------------------------------------------------
%%                                     _     _      
%%      _ __  _ __ ___  __ _ _ __ ___ | |__ | | ___ 
%%     | '_ \| '__/ _ \/ _` | '_ ` _ \| '_ \| |/ _ \
%%     | |_) | | |  __/ (_| | | | | | | |_) | |  __/
%%     | .__/|_|  \___|\__,_|_| |_| |_|_.__/|_|\___|
%%     |_|                                          
%%
%% -----------------------------------------------------------------

\documentclass{kithesis}

% Additional packages
\usepackage[main=english]{babel}

\usepackage{listings}
\usepackage{pythonhighlight}
\usepackage{amsmath}  % for source code
% Listings settings
% See for details: https://en.wikibooks.org/wiki/LaTeX/Source_Code_Listings
\lstset{
    basicstyle=\small\ttfamily,  % smaller typewriter font
    showstringspaces=false       % don't show spaces in string
}
\def\lstlistingname{Zdrojový kód}

% Variables
\thesisspec{figures/zl.pdf}

\title{My thesis \br (the skeleton)}{Customers behavior modeling\\ in e-commerce}

%\author{Janko Hraško}
\author{Aleš}{Jandera}
\supervisor{Tomáš Škovránek} %veduci prace
%\consultant{XXX} %konzultant
%\college{University of Žilina}{Žilinská univerzita} %univerzita
%\faculty{Faculty of Electrical Engineering and informatics}{Fakulta elektrotechniky a informatiky} %fakulta
%\department{Department of Computers and Informatics}{Katedra počítačov a informatiky} %katedra
%\departmentacr{DCI}{KPI} % skratka katedry
%\thesis{Master thesis}{Diplomová práca} %typ prace
\submissiondate{31}{3}{2021}
\fieldofstudy{Extraction and Processing of Earth Resources}
\studyprogramme{Management of Processes}
%\city{Košice} %mesto
\keywords{Matematické modelovanie, modelovanie správania, elektronické obchody}{Mathematical modeling, behavioral modeling, e-commerce}
%\declaration{som nepodvadzal}

\abstract{%
    % slovak
    Táto práca je o matematickom modelovaní a správaní zákazníka. Cieľom je vytvoriť predikčný model, lepšie ako lineárná a polynomická regresia.
    Ako referenčný model sa používa metóda regresnej analýzy a kĺzavého priemeru. V porovnaní s týmito metódami som použil matematický model založený na troch submodeloch a
    zkombinovaný do jedného predikčného mechanizmu skrytým Markovovým modelom. V týchto modeloch sú použité metódy ako Kallmanovo filtrovanie, Viterbiho algoritmus,
    logit analýza, binárne porovnanie a rozhodovacie stromy. Výsledky ukazujú, že lineárna regrési nie je vhodná na presnú predikciu ale velmi dobre odhaduje buducí trend.
    Polynomická regrésia sa jeví lepšia s má priemernú odchýlku okolo 14,45 \%.
}{%
% english
    This thesis is about mathematical modeling and Customer behavior. The goal is to create prediction model in order
    to be better than statistical linear or polynomial regression.
    As a reference model is used regression analysis and move average method.
    In comparison to these methods I used mathematical model based on three submodels combined into one prediction
    mechanism by the hidden Markov model.
    In these models are used methods like Kallman's filtering, Viterbi algorithm, logic analysis, binary comparation and decision trees.
    The results show that linear regression is not suitable for my purpose, but is very usefull in trends prediction.
    Polynomial regression is much better and have average aberrance about 14,45 \%.
}

\acknowledgment{Very strong thanks to whole teaching staff at Institute of Control and Informatization of Production Processes for their patience, leadership and knowledge which help me to write this thesis.
I would like to express my deepest appreciation to my supervisor Tomáš Škovránek for his time and advices.}

% if you want to work only on selected chapters
%\includeonly{chapters/analyza} %,chapters/synteza}

% Load acronyms
\input{acronyms}


%% -----------------------------------------------------------------
%%          _                                       _   
%%       __| | ___   ___ _   _ _ __ ___   ___ _ __ | |_ 
%%      / _` |/ _ \ / __| | | | '_ ` _ \ / _ \ '_ \| __|
%%     | (_| | (_) | (__| |_| | | | | | |  __/ | | | |_ 
%%      \__,_|\___/ \___|\__,_|_| |_| |_|\___|_| |_|\__|
%%                                                      
%% -----------------------------------------------------------------

\begin{document}
%% Title page, abstract, declaration etc.:
\frontmatter{}

%% List of code listings, if you are using package minted
%\listoflistings

%\pagenumbering{arabic}

%% Chapters
% !TEX root = ../thesis.tex

\chaptermark{Introduction}
\phantomsection
\addcontentsline{toc}{chapter}{Introduction}

\chapter*{Introduction}

The goal of my bachelor thesis is to create an Matlab Live script based on assets of the Mathematical models belongs to customer's behavior with mainly focus on e-commerce.
As initial data we will use anonymize data from Megaplay s.r.o stores, both online and offline stores, and open data from
google.com, heureka.sk, heureka.cz and statistical data served by government from years\\ 2018-2019 and then try to predict the income of 1/2020.
As the reference of prediction we will calculate basic statistical methods over that data. Explanation of reference data see in section~\ref{sec:statistics}.
Then we will use behavior models which will be composed especially to simulate customer behavior which is responsible for
buy action with specific point which are important in e-commerce business.\\
As a model demonstration we will used Live script created in Matlab \footnote{MATLAB is a fourth-generation programming
language and numerical analysis environment.
Uses for MATLAB include matrix calculations, developing and running algorithms, creating user interfaces (UI) and data visualization.}
which will have preprocessed income data and all cosntants needed to create prediction of future income
as output with graphical demonstration of prediction.
As a reference method to our models will be used statistics results of Time series and than we compare our model results with basic Time series predictions. For each method we will calculate aberration of the prediction.

\section*{Task formulation}

Creating a mathematical model to predict customer behavior consist of vendor, psychology and loyalty sub-models combine in hidden markov model to final prediction mechanism which should have better results than Time series analyze.

% !TEX root = ../thesis.tex

\chapter{Analytical part}
\label{sec:analytical}
\section{Problem description} \label{sec:description}
Most important thing for each e-commerce project is to predict behavior of their customers or visitors.
In that complex prediction statistics methods are little bit problematic, because of they don't calculate with most of psychology
and sociology aspects.
From statistic methods we can get some basic framework of our prediction income, but in many cases the reality
of the products life can be really different and can crash company business model.
So we are trying to find complex models to predict company income to reflect business plan.
The best approach is the create business plans with that modeling before, so we can minimize the risk of our business.
Two phenomena of particular interest for assessing modelling options are the~\ref{subsec:decoy} and~\ref{subsec:lock-in} principles.
As is written in~\cite{patel}, this modelling is especially used when  the new brand or product is established.
But not in stores on daily basis.
This is caused by the research cost for that studies.\\
\\
\textbf{Decoy effect} \label{subsec:decoy}\\
The Decoy Effect or the Asymmetric Dominance Effect is a cognitive bias in which consumers will tend to have a specific
change in preferences between two vendor options when also presented with a third option that is asymmetrically dominated.
Simply put when there is a third strategically important choice, aka the decoy, then the consumer is more likely to
choose the more expensive of the other two options.
An option is asymmetrically dominated when it is inferior in all respects to one option.
However, in comparison to the other option, it is inferior in some respects and superior in others.
In other words, it is completely dominated by one option and only partially dominated by the other.
When the asymmetrically dominated option is present, a higher percentage of consumers will prefer the dominating
option than when the asymmetrically dominated option is absent.
The asymmetrically dominated option is therefore a decoy serving to increase preference for the dominating option.\\
\\
\textbf{Lock-in} \label{subsec:lock-in}\\
Proprietary lock-in, or customer lock, makes the customer dependent on the products and services of a particular
entity by creating significant costs of switching to the products and services of others.
This can be achieved, for example, by the use of non-standardized patented product components.
Locking, which creates barriers to market entry, can be avoided by antitrust measures.
Proprietary locking is for example blocking mobile phones for only one of the operators or DRM.

\section{Mathematical Background and Marketing introduction} \label{sec:introduction}
Overview of mathematical equation and approaches used to my prediction in context of Marketing approaches usually used to predict customer behavior.\\
\subsection{Models} \label{subsec:models}
\textbf{Simple Dynamic Models} \label{subsec:simpleDynamicModels}\\
One of the simplest models that have been considered for human behavior modeling are dynamic single prices,

\begin{equation} \label{eq:1}
X_k = f(x_k, t) + \xi(t)
\end{equation}
\\
where the function $f$ is dynamic evolution of vector $Xk$ at time $k$.
Than we can define an observation process like

\begin{equation} \label{eq:2}
Y_k = h(X_k, t) + \eta(t)
\end{equation}
\\
\textbf{Multiple Dynamic Models} \label{subsec:multipleDynamicModels}\\
Human behavior is usually not as simple as a single dynamic model.
For complex model of human behavior we can use some alternatives.
In many case we have to use different models for each person's dynamic response~\cite{wilsky}.
Than we can test each instance of response to predict person's state.
From that we can established multiple model approach to predict future values of state variables.
In this situations is Kalman's filter calculation very usefully for realtime application because of it's sufficiently small costs of resources.
This approach breaks the person's overall behavior down into several prototypical behaviors.~\cite{pantland}
Mathematically, this is accomplished by setting up a set of states $S$, each associated with a Kalman's filter and a particular dynamic model.\\
\\
\textbf{A linear ODE model} \label{subsec:ode}\\
In this model the strength of flux between brands or products are determined by perceived brand quality,
based upon binary comparisons.
The simplest response to such comparisons is an attempt by customers to minimize the expected regret resulting
from any choice, which is what is assumed here.
It has previously been shown that the choice rule recognizes the attribute-wise proximity of an alternative
to other brands, and it is therefore appropriate for preference change to be modelled on the pair-wise ranking
of brands in each quality, the simplest perhaps being to assign a positive score to a brand for each successful comparison.
Thus customers attempt to minimize their anticipated regret by opting – on any particular quality – for the safe bet.
More sophisticated customer behavior, capable of not only ranking brands but discriminating according to the size of proximity gap
requires more complex modelling, but may be justified since it appears that subjective attribute valuations at least are nonlinear,
reference-point-dependent functions.\\
\\
\textbf{Discrete-time model} \label{subsec:discrete}\\
Discrete models or difference equations \footnote{Difference equations can be viewed either as a discrete analogue of differential equations, or independently.
They are used for approximation of differential operators, for solving mathematical problems with recurrences, for building various discrete
models, etc.} are used to describe biological events or whole systems for which is natural to regret time at fixed discrete intervals.

\begin{equation} \label{eq:4}
\hat{X}_{k}^{(i)} = X_{k}^{*(i)} + K_k^{(i)}(Y_k - h^{(i)}(X_{k}^{*(i)},t))
\end{equation}
\\
where the superscript $(i)$ denotes the $i$th Kalman's fillter.
The measurement innovations process for the ith model.

\begin{equation} \label{eq:5}
\Gamma_k^{(i)}= Y_k - h^{(i)}(X_k^{*(i)}), t)
\end{equation}
\\
The measurement innovations process is zero-mean with covariance $R$.
The ith measurement innovations process is, intuitively, the part of the observation data that is unexplained by the ith model.
The model that explains the largest portion of the observations is, of course, the model most likely to be correct.
Thus, at each time step, we calculate the probability $Pr^{(i)}$ of the m-dimensional observations $Y_k$ given the $i$th model's dynamics
\\
\begin{equation} \label{eq:6}
Pr^{(i)}(Y_k|X_k^*) = \frac{exp(-\frac{1}{2}\Gamma_k^{(i)T}R^{-1}\Gamma_k^{(i)}}{(2\pi)^{m/2} Det(R)^{1/2}}
\end{equation}
\\
and choose the model with the largest probability.
This model is then used to estimate the current value of the state variables, predict their future values, and choose among alternative responses.

\subsection{Modeling and Prediction of Human Behavior} \label{subsec:prediction}

We have an observations Y as a function h of the state vector and time.
Both $\xi$ and $\eta$ are processes with known density matrices.
Using Kalman's result, we can then find optimal linear estimate $X_k$ using th Kalman's filter \footnote{In statistics
and control theory, Kalman filtering, also known as linear quadratic estimation (LQE), is an algorithm that
uses a series of measurements observed over time, containing statistical noise and other inaccuracies, and produces
estimates of unknown variables that tend to be more accurate than those based on a single measurement alone, by estimating
a joint probability distribution over the variables for each timeframe.
The filter is named after Rudolf E. Kálmán, one of the primary developers of its theory.}

\begin{equation} \label{eq:3}
X = X_{k}^{*} + K_x(Y_k - h(X_{k}^{*},t))
\end{equation}
\\
provided that the Kalman gain matrix $Kk$ is chosen correctly~\cite{kalman}.
This method iterates for each step $k$ and the filter algorithm use a state prediction at each time step $k$,
the filter algorithm uses a state prediction $X$, an error covariance matrix prediction $P_k^*$,
and a sensor measurement $Y_k$ to determine an optimal linear state estimate $X$.
If we want to predict human's future state whe can use this mechanism with larger time steps.
This mechanism is used e.g. in a car, such a prediction capability can allow us to maintain synchrony with the driver.
In experience of Alex Pentland work \footnote{MIT's Human Dynamics Laboratory and the MIT Media Lab Entrepreneurship
Program, co-leads the World Economic Forum Big Data and Personal Data initiatives, and is a founding member
of the Advisory Boards for Nissan, Motorola Mobility, Telefonica, and a variety of start-up firms.}, this type of prediction
is useful only for short time periods, for instance, in the case of quick hand motions for up to one-tenth of a second.
Especially $f$, $h$ are linear functions and $\xi$, $\eta$ are gaussian.
This functions are commonly extended to "well-behaved" nonlinear problems by approximating the nonlinear system by linear functions
using a local Taylor expansion \footnote{Taylor series is a representation of a function as an infinite sum of terms that are calculated from
the values of the function's derivatives at a single point.}.

\subsection{Specification in customer behavior} \label{subsec:specification}
Customer products such as shampoo or tomato sauce are designed to appeal to customers, encouraging them to buy those products.
It depends on the industry section but all of that designs trying to focus to customer's subconscious.
However, buying behavior is not only a function of the product.
In many cases is the connection of many other functions like a social environment of other customers, the competing
products in the marketplace, brand marketing strategy, seller trust and professionalism and so on.
In order to design the best product, it is necessary to understand not just the physics and chemistry of the product,
but also the psychology of customers and the sociology of customer groups or networks.\cite{patel}

From that paragraphs is see that the general human behavior model has many input variables to represent many kind
of situation to be successfully in future prediction.
Good model in store industry have to be able to learn from actual customer's behavior for each seller and learn
from these datasets in a macroscopic, averaged way.
Alternatively, one can look at individual customers and their buying behavior, and try to derive observable large scale effects.
Ideally it can be predict behavior for each customer and from these results create a global results for each store or industry.\\
\\
\textbf{Loyalty} \label{subsec:loyalty}\\
Loyalty is the tendency for some customers to use the same products or brands again.
This behavior we can describe with a systems of ordinary differential equations.
The stronger the loyalty, the slower the changes in numbers of people buying particular products.
For discrete-time models, the degree of loyalty corresponds to the size of diagonal elements in a transition matrix.
In the modelling we have to calculate with the no loyalty model too.
In some industry like a supermarkets, peoples buy from other reason so some of our input variables in modelling should be
detection of loyalty in out industry.
Next aspect of loyalty would be a memory effect, to represent people returning to products they had previously used,
after trying something new they then did not like.
This could be taken into account perhaps by using recurrence relations or differential equations of higher than first order or even employing
delay-differential equations.\\
\\
\textbf{Sociology} \label{subsec:sociology}\\
Mean sociology in this context as how peoples buying are influenced each other.
With some kind of trends to buy the same brands or products.
There is an option from lock-in~\ref{subsec:lock-in}, with one product dominating the market.
This option is very hard to relevant test, because of the data of huge companies which dominates some kind
of industry are very hard to legally get.
Even if theirs competitors have more or less identical products.
This effect and its opposite, are easily modelled by ODE~\ref{subsec:ode} and discrete-time models.
Opposite is very important in sociology because of people wanting to be different sometimes from irrationals reasons.

\subsection{Markov Chains} \label{subsec:chain}
A Markov chain is a process that occurs in a series of time-steps in each of which a random choice is made
among a finite (or also enumerable) number of states; since both the index set and the state space are discrete,
we denote by $X_n≡X(t_n)$ the transition probability can then be represented by a matrix $P=(p_{ij})$, where pij is
the probability of moving from state i to state j: $p_{ij}=Prob[X_{n+1}=j|X_n=i]$. For homogeneous chains, these
probabilities do not depend on t, i.e., they are stationary. Then, the initial distribution, together with the
transition matrix $P$, determines the probability distribution for any state at all future times.\\
\begin{figure}[h!]
	\begin{center}
		\includegraphics[width=140mm]{markov_model.png}
	\end{center}
	\caption{Basic schema of Markov model in human decision}
	\label{Markov model}
\end{figure}
\\
\textbf{A Markov model with social influences} \label{subsec:markov}\\
This model is based on Markov chains (Markov chain description), better than the time continues differential
equations which are especially use.
To use that model approach we first have to develop the possibility of a decoy effect.
Then we introduce sociology and obtain results for lock-in analogous.
These Markov models display both important similarities to and differences from the previous models, and may be simpler to work with.
The last but not least pros of that models is very good graphical representation.\\
\\
\textbf{An Experiment Using Markov Dynamic Models} \label{subsec:markov_dynamic}\\
Shopping is natural-feeling and familiar type of human behavior that exhibits complex patterns that last from several
seconds to many days.
These characteristics make shopping little bit nearly ideal experimental for modeling human behaviors.
In the case of shopping, the macroscopic actions are events like needs, choosing brands, choosing seller.
The internal states are the individual steps that make up the action, and the observed variables will be changes during the shopping process
workflow in which can be customer decision changed.
The intuition is occasionally important, but this belongs to special market and marketing persons identification such
a buying shoes by a single woman in production age.\\
\\
\textbf{Modeling and Prediction of Human Behavior may consist of the following steps:}\\
\begin{enumerate}
	\item needs decision make
	\item looking across the site to find some adequate
	\item make action decision
	\item shopping workflow process in selected site
	\item make final payment
\end{enumerate}
\\
Psychology covers what, and how, aspects of the actual items on the shelves influence people to make their choices,
possibly buying something different from previously.
Advertising might be comprise into these characteristics but could also possibly be considered as part of the sociological
influences (point 1 and 2), especially if the advertising takes the form of a well known figure endorsing a product.
More specifically, the following four properties have been identified by Unilever Research~\cite{patel} as being important
and their influences were included in one or more models:

\begin{enumerate}
	\item Minimise anticipated regret.
	This refers to how just two products compare with each other as regards different qualities, which can include
	price (or 'affordability' = 1/price).
	A customer might judge one item to be superior to another in all respects.
	The first is then a safe choice for the customer.
	\item Attribute change.
	The introduction of a new product onto the market can change the way customers, or at least some of them, view established brands.
	This might be by drawing attention to some quality which was not previously much regarded, or it might make people give different
	weightings to the (established) qualities when making their decisions.
	The former can be considered to be a special case of the latter.
	\item Outlier avoidance.
	When a number of products are in many aspects quite similar, there can be a tendency for people to avoid ‘strange’ ones, like others which
	are substantially different from the majority in price or some other respect.
	Items near the average can be favoured.
	\item Decision process change.
	A straight choice between two items might be relatively easy.
	They can be compared according to price, size etc\. and a decision made.
	With three or more, comparisons might be made between two things at a time, one could be eliminated and then the winner contrasted
	with a third.
\end{enumerate}
\subsection{Viterbi algorithm} \label{subsec:viterbi}
The Viterbi algorithm is a dynamic programming algorithm for finding the most likely sequence of hidden
states called the Viterbi path that results in a sequence of observed events, especially in the
context of Markov information sources and hidden Markov models (HMM)~\ref{Viterbi schema}.
The algorithm has found universal application in decoding the convolutional codes used in
both CDMA and GSM digital cellular, dial-up modems, satellite, deep-space communications, and 802.11 wireless LANs.
It is now also commonly used in speech recognition, speech synthesis, keyword spotting, computational
linguistics, and bioinformatics.
For example, in speech-to-text (speech recognition), the acoustic signal is treated as the observed sequence of events,
and a string of text is considered to be the "hidden cause" of the acoustic signal.
The Viterbi algorithm finds the most likely string of text given the acoustic signal.
Here you can see Viterbi algoritm implementation in Python language: \\
\begin{lstlisting}[language=python]
# probability == p. Tm: the transition matrix. Em: the emission matrix.
function viterbi(O, S, Π, Tm, Em): best_path
  # To hold p. of each state given each observation.
  trellis ← matrix(length(S), length(O))
  # Determine each hidden state's p. at time 0…
  for s in range(length(S)):
    trellis[s, 0] ← Π[s] * Em[s, O[0]]
  # and afterwards, assuming each state's most likely prior state, k.
  for o in range(1, length(O)):
    for s in range(length(S)):
      k ← argmax(k in trellis[k, o-1] * Tm[k, s] * Em[s, o])
      trellis[s, o] ← trellis[k, o-1] * Tm[k, s] * Em[s, o]
  best_path ← list()
  # Backtrack from last observation.
  for o in range(-1, -(length(O)+1), -1):
    # Most likely state at o
    k ← argmax(k in trellis[k, o])
    # is noted for return.
    best_path.insert(0, S[k])
  return best_path
\end{lstlisting}
\begin{figure}[h!]
	\begin{center}
		\includegraphics[width=90mm]{viterbi_graph.png}
	\end{center}
	\caption{Viterbi algorithm schema example}
	\label{Viterbi schema}
\end{figure}
\subsection{Predicted effects} \label{subsec:predicted}
\begin{enumerate}
	\item How a decoy product might influence the market.
	The appearance of a third product might remarkably change the market shares of two others, while getting minimal sales itself.
	This effect is one of the most complex biases in customer choice, and has been observed in product classes from chocolate bars
	to electronics to beer.
	The decoy effect illustrates the importance of customer psychology, to understanding how customers recognize products,
	and how customers see quality side to buy the product.

	\item The dynamics of market share is  how sales of products can change during the time.
	For example, even if two products are really equal in all relevant aspects, then after a long time of customer activity it might be that
	each product takes 50\% market share (preserving the symmetry), or one product takes nearly 100\% market share (breaking the symmetry),
	or that there is no steady state, with market dominance alternating between the two brands.
	The second of these three cases is called lock-in, corresponding to one brand obtaining a virtual monopoly,
	which is almost impossible to break.
	From that reason is not legally to create a monopole in some industries and for human purchase behavior is not
	good to calculate with monopole approach.

	\item How a new product will fare, given its quality profile compared with existing brands.
	This question is complementary to that of the decoy, asking what market share a new product will gain rather than how it will
	affect the market shares of existing products.

	\item Choice overload: when there are just too many possible options for potential customers to pick from, and many
	will search the sites without making a purchase.
\end{enumerate}

\subsection{Minimise anticipated regret} \label{subsec:regret}
This modelling property was taken to lead to simple comparisons along the lines of with regard to quality $k$,
is product $i$ better than product $j$?
If the answer to all relevant questions, $k = 1, . . . , nq$, if $nq$ is the number of qualities, is no, then a customer
will not change from $j$ to $i$.
The more times the answer is yes, the faster such a change is likely to happen.
This can be represented by taking the flow-rate constants to be of the form (todo equation)

\section{Customer preferences and decision process} \label{sec:customer_preferences}
Consider a customer whose preference is shared out amongst all the available brands in a market where there are
no empty or zero brands, so that the total of all brands preference shares is 1 (100\%).
The proportion of customer preference held by brand $X$ at time $t$ is denoted by $X(t)$.
For example, if the preference share of brand $A$ is plotted against brand $B$ in a market where only two brands exists,
the point must be somewhere along the straight line $B(t) = 1, A(t)$.
Of interest is the case when a third brand is added, possibly as a decoy.
The additional brand means that the preference distribution changes from being a straight line in the two-dimensional plane $(A, B)$,
to a plane in three-dimensional space $(A,B,C)$.\\
\\
\textbf{Customer decision process} \label{sec:decision}\\
Described in~\cite{patel} show that the standard Logit model~\ref{subsec:logit} for customer choice assumes that
the probability, $p_i$, with which a customer buys a given product $i$ from a range of products $<1, n>$ depends on
the value which customer attaches to that product $V_i$, and the price of the product $P_i$.
This dependence is taken to be of exponential form.
Assuming that guaranteeing that all probabilities are positive)
\\
\begin{equation} \label{eq:7}
p_i = C exp(V_i - P_i)
\end{equation}
\\
where $s$ is a measure of the relative importance of price to the customer, and $C$ is a normalisation constant chosen so that
\\
\begin{equation} \label{eq:8}
\sum_{i=1}^n p_i = 1
\end{equation}
\\
The value $V_i$ is then taken to reflect the influence on the customer of the quality of the product,$Q_i$,
the increased likelihood of the customer buying the same brand as he bought previously (the loyalty effect),
and the peoples who have an influence on a customer (neighbours).
Each of these dependencies are taken to be linear, giving
\\
\begin{equation} \label{eq:9}
V_i = aQ_i + lI_i + hN_i,
\end{equation}
\\
where $I_i$ is an indicator function which is unity if the customer previously bought product i and zero otherwise,
$N_i$ is the number of neighbours who bought product $i$, and $a, l$ and $h$ are constants measuring the relative strength
of each effect.
In such a model the products are all treated independently.
Only coupling between the probabilities occurs through the normalisation constant $C$.
Product $I$ will depend not only on the value and price of that product, $V_i$ and $P_i$, but on the value and prices of all products.
The goal of this section is to formulate a model for this probability which is based on binary comparisons, that is,
on comparisons of two products at a time.\\
\subsection{Logit analysis in marketing} \label{subsec:logit}
Logit analysis is a statistical technique used by marketers to assess the scope of customer acceptance of a product, particularly a new product.
It attempts to determine the intensity or magnitude of customers' purchase intentions and translates that into a measure of actual buying behavior.
Logit analysis assumes that an unmet need in the marketplace has already been detected, and that the product has been designed to meet that need.
The purpose of logit analysis is to quantify the potential sales of that product.
It takes survey data on consumers purchase intentions and converts it into actual purchase probabilities.
Logit analysis defines the functional relationship between stated purchase intentions and preferences, and the actual probability of purchase.
A preference regression is performed on the survey data.
This is then modified with actual historical observations of purchase behavior.
The resultant functional relationship defines purchase probability.

\subsection{Brand or product changing} \label{subsec:brand}
We can mathematically express the process of decision to switch from one brand to another one.
Consider a linear flux $\alpha_{xy}$ of preferences moving to brand $X$ from brand $Y$.
All fluxes have to be strictly positive.
It's in improper to consider negative flux, so really excluded is only zero value.
Flux is the proportion of customer preference to switch to other brand or product from previous one.
In a two brand or two producst market we have these final equations

\begin{equation} \label{eq:10}
\begin{array}{r@{}l}
\frac{dA}{dt} = -\alpha_{BA}A + \alpha_{AB}A \\
\frac{dB}{dt} = +\alpha_{BA}A - \alpha_{AB}A \\
A(t) + B(t) = 1
\end{array}
\end{equation}

\subsection{New brand/product on the market} \label{subsec:newbrand}
Equation from the previous point ve can easily extend in the situation like a new product or brand is enter
stable market and change customer behavior.

\begin{equation} \label{eq:11}
\begin{array}{r@{}l}
\frac{dA}{dt} = -(\alpha_{BA}+ \alpha_{CA})A + \alpha_{AB}A + \alpha_{AC}C \\
\frac{dB}{dt} = +\alpha_{BA}A - (\alpha_{AB} + \alpha_{CB})B + \alpha_{BC}C \\
\frac{dC}{dt} = +\alpha_{CA}A + \alpha_{CB}B - (\alpha_{AC} + \alpha_{BC})C \\
A(t) + B(t) + C(t) = 1
\end{array}
\end{equation}

\section{Networks} \label{sec:networks}
In this section we consider the propagation of consumer behaviour across a square lattice
(generalizations to other geometries are straightforward) in which each node represents an individual.
Each node/individual, denoted by the pair (I,J), makes a choice between a number of products depending on
the products’ properties and on the views or opinions the ‘preceding’ nodes, i.e. those that lie below
or to the left on the square lattice. We consider two or three products, characterised by their qualities,¨
such as affordability. As an individual, a node has a psychology and a sociology, which determine its
perception of each product. These can be modelled with various degrees of sophistication.\\
\\
\textbf{Decision tree} \label{subsec:tree}\\
To see how such a model gives rise to the probabilities of choosing each product we need to
construct a decision tree. In forming the decision tree we have to decide which two  products the consumer
will choose to compare. We assume initially for simplicity that this decision is uniformly random, so that
there is an equal probability of choosing each pair. We also have to decide how many comparisons the consumer
will make, that is, having chosen a winner between the first two products, the consumer may then compare
this winner with another product, and so on. Each level of the decision tree will have two stages, the choice
of which products to compare, and the outcome of the comparison.\\
\\
\textbf{Level 1 decision tree}\\
The simplest decision tree for the three products shown in Fig. 10, with one level of
comparison (that is, the consumer stops after comparing just two products) is shown in Fig. 11.
Let us assume that products 1 and 2 are equally attractive to the consumer, so that in the absence of
product 3 each would have a probability of selection of 1/2. We can then see what the introduction of
product 3 does to this balance of probabilities, and in particular whether there is a decoy effect,
that is, since product 1 dominates product 3, the introduction of product 3 might mean that more people
will choose to buy product 1 (since product 1 will win in any comparison between the two). In this scenario
product 3 acts as a decoy, channelling consumers to product 1. Let us assume that the probability of choosing
product 3 over product 2 in a comparison is p.\\
The probability of reaching any leaf in the decision tree is simply the product of the probabilities of
taking each branch required to get there    . Thus, after introducing product 3 the probabilities of choosing each product are\\
\\
\begin{equation} \label{eq:12}
\begin{array}{l@{}l}
p_1 = \frac{1}{3}*\frac{1}{2}+\frac{1}{3}*1 = \frac{1}{2} \\
p_2 = \frac{1}{3}*\frac{1}{2}+\frac{1}{3}*(1-p) = \frac{1}{2} - \frac{p}{3} \\
p_3 = \frac{1}{3}*0+\frac{1}{3}*p = \frac{p}{3}
\end{array}
\end{equation}
\\
Thus we see that in this simple model there is no decoy effect; the probability of choosing product 2 has decreased,
but the probability of choosing product 1 is the same as it was before product 3 was introduced.
This (perhaps surprising) result occurs because although there is now a proportion of binary comparisons
that product 1 is guaranteed to win (those between products 1 and 3), there is also a proportion of comparisons
that do not involve product 1 at all (those between products 2 and 3). Thus there is a chance that the consumer
does not even choose to examine product 1, and this exactly offsets the effect of the decoy.
A simple modification of this model can be used to consider the effect of a decoy on product 1 when
there is more than one competitor. Suppose that instead of one competitor (product 2) there are n competitors
to product 1, and that all the competitors are equivalent from the point of view of the consumer.
Then we can lump all the competitors into a single product number 2, with the main change to the decision tree
being to the chance of choosing products to compare. Of course it is now possible that the consumer
chooses to compare two competitors products, and we must take this into account. The new decision tree is shown in Fig. 12.
The probabilities of choosing each product are
\\
\begin{equation} \label{eq:13}
\begin{array}{l@{}l}
p_1 = \frac{2n}{(n+2)(n+1)}*\frac{1}{2}+\frac{2}{(n+2)(n+1)}*1 = \frac{1}{(n+1)} \\
p_2 = \frac{2n}{(n+2)(n+1)}*\frac{1}{2}+\frac{2n}{(n+2)(n+1)}*(1-p) + \frac{n(n-1)}{(n+2)(n+1)} = \frac{n}{n+1} - \frac{2np}{(n+2)(n+1)} \\
p_3 = \frac{2np}{(n+2)(n+1)}
\end{array}
\end{equation}
\\
We see that there is no decoy effect. Product 1 has exactly the same market share as it would have if product 3 were not present.\\
\begin{figure}[h!]
	\begin{center}
		\includegraphics[width=100mm]{level1DT.png}
	\end{center}
	\caption{Schema of decision tree function}
	\label{Decision tree level 1}
\end{figure}
\\
\textbf{Level 2 decision tree}\\
Let us now go back to just three products, and assume that the consumer does not stop at one level of comparison,
but takes the winner of the first comparison and compares it with the remaining product. In this case
we have the level 2 decision tree shown in Fig. 13. Note that for the second comparison there is no
choice of products to compare, since there is only one product remaining. The probabilities of choosing each product are now\\
\begin{equation} \label{eq:14}
\begin{array}{l@{}l}
p_1 = \frac{1}{6}+\frac{1}{6}+\frac{p}{3}+\frac{1-p}{6}=\frac{1}{2}+\frac{p}{6} \\
p_2 = \frac{1-p}{6}+\frac{1}{6}+\frac{1-p}{6}=\frac{1}{2}-\frac{p}{3} \\
p_3 = \frac{p}{3}
\end{array}
\end{equation}
\\
Thus in this case the market share of product 1 does increase, so that there is a decoy effect.
Note, however, that the market share of the decoy product number 3, is the same as the increased market share of product 1.
\\
\begin{figure}[h!]
	\begin{center}
		\includegraphics[height=100mm]{level2DT.png}
	\end{center}
	\caption{Basic schema of decision tree level 2}
	\label{Decision tree level 2}
\end{figure}
\newpage
\subsection{Propagation thrue a network} \label{subsec:propagation}
We consider here a simple linear model. Each node (I,J) is given, as psychology, a sensibility κkIJ to each quality k.
Each node, as loyalty, is influenced by the preceding nodes through a coefficient λIJ.
We have taken the interaction between the nodes so that (I, J) is affected only by its two immediate neighbours,
(I − 1, J) and (I, J − 1), and both exert the same influence per individual (the area of influence is easy to generalize).
In our simulations, the coefficients κkIJ and λIJ are positive and randomly chosen from a uniform distribution.
Consumer (I,J) forms an opinion or view, in the form of a composite numerical value, on product i according to
$$V_{IJ}^{i} = \sum_k K_{IJ}^kQ_{ki} + \lambda_{IJ}(V_{I-1,J}^1+V_{I,J-1}^i), for I > 0, J > 0$$
\\
\textbf{Binary comparisons} \label{subsec:binary}\\
Suppose a customer chooses to compare product i with product $j$.
What is the probability that he will choose $i$ over $j$?
In the Logit model we would simply have
\\
\begin{equation} \label{eq:15}
p_i = C exp(V_i - P_i), p_j = C exp(V_j - P_j)
\end{equation}
\\
where $C$ is chosen so that $p_i + p_j = 1$.

\begin{figure}[h!]
	\begin{center}
		\includegraphics[width=100mm]{affordability.png}
	\end{center}
	\caption{Products in the affordability-value plane.
	The shaded region is the region of products dominated by product 1.}
	\label{Affordability of products}
\end{figure}

\section{A particle-dynamics model} \label{sec:dynamic}
The basic idea from~\cite{patel} is to build a model based on a physical analogy, between the customers buying behavior
and particle-particle dynamics.
We assume customers to be particles moving in quality space.
\begin{figure}[h!]
	\begin{center}
		\includegraphics[width=80mm]{strength.png}
	\end{center}
	\caption{Product space with two products}
	\label{Strength of products}
\end{figure}
\\
The products are considered as sources of strength potentials, the details of the potentials depending on the products’ characteristics.
The psychology of people will be regarded as the influence of the potential on the mass of the particles.
This will depend on the customer but also on the characteristics of each product, weighted by coefficients of choice.
The sociology of individuals can be considered as a form of particle-particle interaction.
Following what has gone before, the product space is defined by just two characteristics for each product $i$ with affordability $P_i$ and quality $Q_i$.
We can define the ‘strength’ \footnote{ thw power if the product or service to be attractive for customer} of a product as the distance
of the product from the origin in product space as
\\
\begin{equation} \label{eq:16}
Ui = (Pi + Qi)/2
\end{equation}
\\
All the products with $Pi + Qi = constant$ will have the same strength.

\subsection{Dynamics of market share} \label{subsec:marketshare}
Market shares don't unusually depend in their qualitative behavior, on the coefficients $\alpha_{ij}$ and the values
of $\alpha_{ij}$ grow up as fast as their tend towards stability values.
In the situation  that $\alpha_{ij} \geq 0$ and $\alpha_{ij} + \alpha_{ji} > 0$ for $i \neq j$ the linear system looks like
\\
\begin{equation} \label{eq:17}
\frac{dX_i}{dt} = \sum_ja_{ij}X_j
\end{equation}
\\
where $a_{ij} = \alpha_{ij}$ for $i \neq j$ and $a_{ii} - \sum_{i \neq j}\alpha_{ji}$ turns out to have semi-definite
coefficient matrix.

\section{Statistic solutions of data prediction} \label{sec:statistics}
As a reference method for model result compare will be used time series prediction.
Time series data often arise when monitoring industrial processes or tracking corporate business metrics.
The essential difference between modeling data via time series methods or using the process monitoring methods.
Time series analysis accounts for the fact that data points taken over time may have an internal structure such as autocorrelation,
trend or seasonal variation that should be accounted for.\\
\\
\textbf{Analyze of time series} \label{subsec:statistics_analyze}\\
As is written by~\cite{benkova} we can divide time series to
\begin{itemize}
	\item subjective and objective
	\item quantitative, qualitative and causal
\end{itemize}\\
\\
\textbf{As basic characteristics we can define:}
\begin{itemize}
	\item absolute growth (1.difference)
	\begin{equation} \label{eq:18}
	\Delta_t^1 = y_t - y_{t-1}, t = 2,3 ....,n
	\end{equation}
	\item relative growth
	\begin{equation} \label{eq:19}
	\delta_1 = \frac{\Delta_t^1}{\Delta_{t-1}^1}, t = 2,3 ....,n
	\end{equation}
	\item growth coefficient
	\begin{equation} \label{eq:20}
	k_t = \frac{y_t}{y_{t-1}}, t = 2,3 ....,n
	\end{equation}
\end{itemize}\\
\\
Modeling of time series can be done by one-dimensional or two-dimensional models.
Basic one-dimensional model is used in situation where expected that time will be the only one variable.\\
\\
\textbf{Classical model of time series have these components:}
\begin{itemize}
	\item $T_t$ trends component, characteristic of long term trend
	\item $C_t$ cyclical component, fluctuation around trend
	\item $S_t$ seasonal component, periodical component repeated in time
	\item $I_t$ irregular component, random component, stay in time series after all corrections
\end{itemize}
\\
For pur need we will be used only Linear $T_t = a_0 = \overset{-}{y}, t = 1,2, ...., n$ and exponential trend
$T_t = a_{0}a_1^t, t = 1,2,....,n, a_1 > 0$.
\\
For testing which model is better for our test scenario we use Fisher test \footnote{Fisher's exact test is a statistical test
used to determine if there are nonrandom associations between two categorical variables.
For each one, calculate the associated conditional probability using, where the sum of these probabilities must be 1.} of local extrema
in periodogram and Durbin Watson test \footnote{The Durbin Watson statistic is a test for autocorrelation in the residuals
from a statistical regression analysis.
The Durbin-Watson statistic will always have a value between 0 and 4.
A value of 2.0 means that there is no autocorrelation detected in the sample.} of autocorrelation.\\
\\
\textbf{Forecasting with Single Exponential Smoothing} \label{subsec:statistics_forecast}
\\
The forecasting formula is the basic equation
\begin{equation} \label{eq:21}
S_t+1 = \alpha y_t+(1−\alpha)S_t, 0< \alpha \leq 1,t>0
\end{equation}
\\
This can be written as:
\begin{equation} \label{eq:22}
S_t+1=S_t+\alpha \epsilon_t
\end{equation}
\\
where $\epsilon_t$ is the forecast error (actual - forecast) for period $t$.
In other words, the new forecast is the old one plus an adjustment for the error that occurred in the last forecast.\\
\\
\textbf{Bootstrapping of Forecasts}\\
If you wish to forecast from some origin, usually the last data point, and no actual observations are available
we have to modify the formula to become:
\\
\begin{equation} \label{eq:23}
S_t+1 = \alpha y_{org}+(1−\alpha)S_t
\end{equation}
\\
where $y_{orig}$ remains constant.

% !TEX root = ../thesis.tex

\chapter{Syntactic part} \label{sec:methodology}
Based on the Analytical part~\ref{sec:analytical} let us create a mathematical models to predict customer behavior consist of
vendor, psychology and loyalty sub-models (Described in own section ~\ref{sec:submodels}) combined in Hidden markov model (HMM) to final prediction model.
Hidden states in the HMM produced in readable format by viterbi algoritm (See section Viterbi~\ref{subsec:viterbi}) we used for orders calculation for final income prediction.
Viterbi serve us the state with the most probability during propagation true HMM.
We lead up to get a better results than actual standard statistical regression methods~\ref{sec:regression} to predict customer behavior in e-commerce.
In the final our model will return future income for online store based on previous data with better aberration than linear or polynomial regression has.
For successfully prediction open data for store strength and customer satisfaction and some predefined and computed variables from store will be used.
\section{Modeling prediction of customer behavior} \label{sec:modeling}
Let us the model to store information about successfully order prediction.
In the same way the model is able to predict where the customer leaves the store, but this situation was not added to our prediction experiment~\ref{experiment}
Then based on that prediction data, calculation of future income for store were done.
Our designed model is consists of three sub-models, which are partially dependent on each other.
Vendor ,\ psychology and loyalty models we created for our prediction to simulate customer behavior.
Then partial results was combined to complex prediction mechanism, as you can see on figure~\ref{Model schema with interaction}.
As a combination method Hidden Markov model was selected as appropriate.
HMM in combination with Viterbi algorithm will be used to detect hidden states, which will be reflected predefined weights for specific industries,
to return data matrix with prediction.
Industry dependent weights was set with Megaplay s.r.o employees, based on their data and experiences.
\\
\begin{figure}[h!]
    \begin{center}
        \includegraphics[width=140mm]{computation_schema.png}
    \end{center}
    \caption{Computation flow and interactions between models}
    \label{Model schema with interaction}
\end{figure}
\\
\subsection{Decision process from sub-models (Hidden markov model)} \label{sec:decision}
In a Hidden Markov Model (HMM), we have an invisible Markov chain (which we cannot observe), and each state
generates in random one out of $k$ observations, which are visible to us.
Let’s look at an example. Suppose we have the Markov Chain from above, with three states (snow, rain and sunshine),
$P$ - the transition probability matrix and $q$ — the initial probabilities.
This is the invisible Markov Chain — suppose we are home and cannot see the weather.
We can, however, feel the temperature inside our room, and suppose there are two possible observations: hot and cold.\\
Let us go to our situation.
For our need we will create 3 states model.
States will be:\\
\begin{itemize}
    \item order
    \item not finished order
    \item no order decision
\end{itemize}
With Megaplay s.r.o owner and heureka e-comerce tool calculation data we were set probability wages vector for HMM function like:\\
$$ p_w = \left(\frac{1}{3} & \frac{1}{2} & \frac{1}{9}\right) $$
\\
This vector tells about probabilities for each submodel described in own section~\ref{sec:submodels}
Vector have to be updated to square matrix which is used in HMM function.\\
\begin{equation*}
    P_w =
    \begin{pmatrix}
        \frac{1}{3} & \frac{1}{2} & \frac{1}{9} \\
        \frac{1}{3} & \frac{1}{2} & \frac{1}{9} \\
        \frac{1}{3} & \frac{1}{2} & \frac{1}{9}
    \end{pmatrix}
\end{equation*}\\

\subsection{Preprocessing of input data} \label{subsec:preprocessing}
Data should have to be anonymized \footnote{Data anonymization is a type of information sanitization whose intent is privacy protection.
It is the process of either encrypting or removing personally identifiable information from data sets, so that
the people whom the data describe remain anonymous.} and pseudonymized \footnote{Pseudonymization is a data management
and de-identification procedure by which personally identifiable information fields within a data record are replaced
by one or more artificial identifiers, or pseudonyms.} to keep legal notice of~\cite{gdpr}.
Then we should utilize data to utilized inputs.
This prepared data I will add to the Matlab live script file to be used for prediction.\\
\\
\textbf{Average day visitors for a predicted month}\\
This value was calculated from anonymize user data from Google Analytics tool used their prediction mechanism to get number of future users based on previous a number of visitors.\\
\\
\textbf{Perceived value for psychology model} \label{perceived}\\
This variable is needed for loyalty model~\ref{subsec:model_loyalty} and comes from an open e-commerce compare data provided by Heureka.cz/Heureka.sk internal tool.\\
\\
\textbf{Number of orders 2018 - 2019}\\
Number of orders get from shopycrm.com tool used in Megaplay s.r.o to manage their business processes and store all needed data.\\
\\
\textbf{Unique products sell}\\
This variable is used in Psychology model~\ref{subsec:model_psychology} for Price aspect calculation~\ref{eq:26}.\\
\\
\textbf{Customer satisfaction} \label{customerSat}\\
This value is provided by Heureka open data, and it's a calculated value from the all customer reviews of the store.\\
\\
\textbf{Margin}\\
The retail margin percentage measures the retail margin as a percentage of the retail price.
This measurement gives you a context for the retail margin.
For example, if you have a 5~€ retail margin on two different products, but one costs 150~€ and next one costs 10~€, the second product would have a much higher retail margin percentage.\\
\\
\textbf{Number of product order each day $Q_i$}\\
This is a calculated value from shopycrm.com about number of product order for each one.
It's used in Psychology model~\ref{subsec:model_psychology}. \\
\\
\textbf{Quality index $Q_1$}\\
This is vendor quality index provided by Heureka open data.
This is a power/strength of the store.
Higher number mean that for a customer is more difficult to leave the store and go to another online store.\\
\\
\textbf{Vendor coefficients $\beta, \gamma, \delta$} \label{vendorCoeff}\\
This three coefficients are used as weight for vendor model~\ref{subsec:model_vendors} and was set with cooperation with Megaplay s.r.o owner for their industry.
It's represent the relation between vendor prices and vendor quality.\\
\\
\begin{table}[h!]
    \begin{center}
        \begin{tabular}{ | l | c |}
            \hline
            {\textbf{Variable name}} & \textbf{Result}\\
            \hline
            average day visitors for a predicted month $U_a$& 2357 \\
            perceived value for psychology model & 0.87 \\
            number of orders 2018 - 2019 $O_c$ & 26 530 \\
            Unique products sell $U_p$ & 11048\\
            Customer satisfaction & 90\%\\
            Margin & 0.27\\
            $Q_i$ & 4.12\\
            $Q_1$ & 0.94\\
            Vendor coefficients: & \\
            $\beta$ & 0.7\\
            $\gamma$ & 0.7\\
            $\delta$ & 0.85\\
            \hline
        \end{tabular}
    \end{center}
    \caption{Coefficient data from Megaplay s.r.o}
    \label{megaplay_data}
\end{table}
\\
\begin{table}[h!]
    \begin{center}
        \begin{tabular}{ | l | c |}
            \hline
            {\textbf{Variable name}} & \textbf{Equation}\\
            \hline
            number of visitors 1/2020 & $31 * U_a$\\
            average earn per order & $\sum Income / O_c$\\
            \overline{Q} & $1/U_p * (U_p/ O_c)$\\
            \overline{P} & $1/U_p * margin$\\
            trust & $1/O_c * O_c/U_a$\\
            \hline
        \end{tabular}
    \end{center}
    \caption{Calculated data from Megaplay s.r.o}
    \label{megaplay_data_equation}
\end{table}
\\
\section{Submodels for Customer behavior model} \label{sec:submodels}
\subsection{Vendors} \label{subsec:model_vendors}
For simplification, we will take to account only a two vendors market share, without 100\% domination, because of on a real
market is 100\% monopole unlikely see and then we will be reflected only main competitor.
From that condition it is evident that for market dominate vendor it will return false positive results.
Our model is based on equations from Brands section~\ref{eq:10} and from those equations we will get probability of vendor change from
vendor $A$ to vendor $B$ and if the vendor $B$ will have more strength than vendor $A$ it will break the computation and customer
will leave our store without successfully completing buy process.
Results of this computation wil be probability for each iteration of buy process simulation in an interval $<0,1>$.
Data for this model will comes from open data provided by google.com, heureka.cz/heureka.sk and national governments.
That data will be combined with calculated price indexes downloaded from shopycrm.com tool.
\\
\begin{equation} \label{eq:24}
\alpha_{xy} = \beta H(P_x-P_y) + \gamma H(Qx-Qy) + \delta H(Px-Py)H(Q_x - Q_y)
\end{equation}
\\
where $\beta, \gamma, \delta$ are non-negative~\ref{vendorCoeff}.
\\
\begin{itemize}
    \item $P$ is prices of vendors products
    \item $Q$ is a quality index of vendor
    \item $H$ is a Heaviside function which will be calculated by~\ref{eq:24}.
\end{itemize}
\\
\begin{eqnarray} \label{eq:25}
H(s) = 1, s > 0 \\
H (s) = 0, s \leq 0
\end{eqnarray}
\\
\subsection{Psychology} \label{subsec:model_psychology}
Psychology aspect is trying to simulate customer behavior in thee situation like a prices aspect, society influenced, mood aspect, actual needs
and so on.
In this model we will simplify only for price effect~\ref{subsubsec:model_psychology_price} and center of mass effect~\ref{subsubsec:model_psychology_mass}
\subsubsection{Price aspect} \label{subsubsec:model_psychology_price}
\begin{equation} \label{eq:26}
\overset{-}{Q} = \frac{1}{n_p} \sum_{i=1}^{n_p} Q_i
\end{equation}
\\
\begin{itemize}
    \item $Q_i$ number of orders for product $i$ per day divide amount of orders per same day, prom interval $<0,1>$
    \item $n_p$ number of products in store
\end{itemize}
\\
\begin{equation} \label{eq:27}
\alpha_{ij} = \frac{C+max(Q_j, \overset{-}{Q})}{C+max(Q_i, \overset{-}{Q})}
\end{equation}
\\
Coefficient C is used to be $\alpha_{ij}$ always positive.
\subsubsection{Center of mass aspect} \label{subsubsec:model_psychology_mass}
Center of mass aspect is applied as a part of sociology to our psyhology model.
Marketers use it to manipulate with customers in global way.
Like a black friday, Cyber monday etc., in that days stores manipulate with our psychology by discount prices.
\\
\begin{equation} \label{eq:28}
\overset{-}{P} = \frac{1}{n_p} \sum_{i=1}^{n_p} P_i
\end{equation}
\begin{itemize}
    \item $P_i$ is defined as product price minus retail recommend price divide average price
    \item $n_p$ number of products in store
\end{itemize}
\\
\begin{equation} \label{eq:29}
\alpha_{ij} = \frac{C+max(P_j, \overset{-}{P})}{C+max(P_i, \overset{-}{P})}
\end{equation}
\\
This model returns probability for customer decision to make action from interval $<0,1>$.
\\
\subsection{Loyalty} \label{subsec:model_loyalty}
This models is based on Luarn \& Lin research~\cite{luarn} and we change theirs model for our needs.\\
\begin{equation} \label{eq:30}
L = \frac{R+Z}{Z}
\end{equation}
\\
\begin{enumerate}
    \item L ..... probability of whole loyalty model
    \item R ..... probability of separate loyalty model
    \item Z ..... probability of commitment model
\end{enumerate}
\\
As we see on figure~\ref{Loyalty scheme} Loyalty model is consist of all three sub-models (Trust, Customer Satisfaction, Perceived value)
but commitment model consist of only Trust and Customer Satisfaction.
a,b,c,d,e .... weight coefficient to combine Trust, Customer satisfaction and Perceived value.\\
\newpage
Trust
\begin{equation} \label{eq:31}
T = \frac{1}{n} \sum_{i=1}^{n} O_i
\end{equation}
\\
$O_i$ number of orders divide number of visitors per day $i$
\\
\\
\textbf{Customer satisfaction} (Se defined values in~\ref{customerSat}) are datasets get from open data per each day, individual satisfaction.
For simplified the situation we will use average satisfaction per whole store.
\textbf{Perceived value} (Se defined values in~\ref{perceived}) is subjective value, which depend on a strength of the store.
\subsubsection{Commitment and Loyalty} \label{subsubsec:model_loyalty_commitment}
\textbf{Commitment} is making the actual choice every day or other period basis to keep up with something e.g. a relationship, personal goal, a task, etc.
We would have to hold ourselves accountable to keep a commitment to something or someone.
Similarly, being loyal involves holding ourself accountable as well.
Against of \textbf{Loyalty} which is usually seen as a character trait rather than a conscious decision.
This model return probability for customer decision to not leave the store from interval $<0,1>$.\\
\\
\begin{figure}[h!]
    \begin{center}
        \includegraphics[width=120mm]{loyalty.png}
    \end{center}
    \caption{Computation flow for loyalty and commitment~\cite{luarn}}
    \label{Loyalty scheme}
\end{figure}\\
\newpage
\subsection{Combining models to decision process} \label{subsec:combining_models}
All separates models return probabilities vectors which we update to square matrix which is use as input to Hidden Markov model (HMM) \footnote{Hidden Markov Model (HMM) is a statistical Markov model
in which the system being modeled is assumed to be a Markov process with unobservable (i.e. hidden) states.
The hidden Markov model can be represented as the simplest dynamic Bayesian network.
The mathematics behind the HMM were developed by L. E. Baum and coworkers.
HMM is closely related to earlier work on the optimal nonlinear filtering problem by Ruslan L. Stratonovich,
who was the first to describe the forward-backward procedure.}
These probabilities will be combined with specific weights coefficients to always prevent positives results.
\\
\section{Experiment} \label{experiment}
Let us prepare experiment to test our models. We were prepared three models.
Linear regresion for model:\\
\begin{equation} \label{eq:40}
f(x) = a(\sin(x-\pi))+b((x-10)^2)+c*(1),
\end{equation}\\
with coefficients $a =2.978.10^5,b =1687,c = 1.753.10^6$, polynomial regression for model:\\
\begin{equation} \label{eq:41}
f(x) = p1*x^4 + p2*x^3 + p3*x^2 + p4*x + p5,
\end{equation}\\
with these coefficients: $p1 = 225.1, p2 = -1.059.10^4, p3 =1.596.10^5,\\
p4 = -8.205.10^5, p5 = 2.702.10^6$ and our Customer behavior model, described in figure~\ref{Model schema with interaction}.
These model get us prediction for year 2020.
Prediction period we compared and calculate aberration for each one against real store income from 2020.
Experiment was simulated in Matlab, where internal predefined function was used. See more in section~\ref{subsec:libraries}.
Let see our results in Evaluation section~\ref{evaluation} and Summary section~\ref{summary}.\\
\subsection{Used software, libraries and predefined functions} \label{subsec:libraries}
\textbf{Matlab 2020a}\\
MATLAB (matrix laboratory) is a fourth-generation high-level programming language and interactive environment for numerical
computation, visualization and programming developed by MathWorks.\\
\\
\textbf{Matlab LiveScript}~\cite{livescript}\\
MATLAB live scripts and live functions are interactive documents that combine MATLAB code with formatted text, equations,
and images in a single environment called the Live Editor.
In addition, live scripts store and display output alongside the code that creates it.\\
Use live scripts and functions to:\\
\begin{itemize}
    \item Visually explore and analyze problems
    \item Share richly formatted, executable narratives
    \item Create interactive lectures for teaching
\end{itemize}\\
\\
\textbf{Matlab Curve Fitting Tool}\\
The Curve Fitting Toolbox provides a collection of GUIs and M-files.
With the toolbox we are able to data preprocessing such as sectioning and smoothing, parametric and nonparametric data fitting,
fit statistics to assist you in determining the goodness of fit,analysis capabilities such as extrapolation, differentiation, and integration
A graphical environment that allows you to explore and analyze data sets and fits visually and numerically.\\
\textbf{hhmgenerate}~\cite{hhmgenerate}\\
The function $hmmgenerate$ begins with the model in state 1 at step 0, prior to the first emission.
The model then makes a transition to state $i_1$, with probability $T_{1i_1}$, and generates an emission $a_k_1$ with probability $E_{i_1k_11}$.
$hmmgenerate$ returns $i_1$ as the first entry of states, and $a_k_1$ as the first entry of seq.
$[seq,states] = hmmgenerate(len,TRANS,EMIS)$ takes a known Markov model, specified by transition probability matrix $TRANS$ and emission probability matrix $EMIS$,
and uses it to generate:\\
\begin{itemize}
    \item A random sequence seq of emission symbols
    \item A random sequence states of states
\end{itemize}
The length of both $seq$ and $states$ is $len$.
$TRANS(i,j)$ is the probability of transition from state $i$ to state $j$.
$EMIS(k,l)$ is the probability that symbol $l$ is emitted from state $k$.\\
\\
\textbf{hhmviterbi}~\cite{hhmviterbi}\\
The function $hmmviterbi$ begins with the model in state 1 at step 0, prior to the first emission.
hmmviterbi computes the most likely path based on the fact that the model begins in state 1.
$STATES = hmmviterbi(seq,TRANS,EMIS)$ given a $sequence, seq$, calculates the most likely path through the hidden Markov model
specified by transition probability matrix, $TRANS$, and emission probability matrix $EMIS$. $TRANS(i,j)$ is the probability of transition from state $i$ to state $j$.
$EMIS(i,k)$ is the probability that symbol $k$ is emitted from state $i$.\\
\\
\textbf{shopycrm.com}\\
Online CRM application focused on e-commerce stores which provides all store workflows and get precalculated data which we will use for our models to simplify the prediction calculation.
\newpage
\subsection{Linear and Polynomial regression} \label{sec:baseline}
Let use Matlab Curve fitting tool to solve our linear and polynomial models as you can see on figure~\ref{cftool}, this predefined tool calculate coefficient for and then
the prediction in Matlab Live script was done as we see on figure~\ref{predict}
These two models was trained on income data from years 2018 and 2019.
\begin{figure}[h!]
    \begin{center}
        \includegraphics[width=90mm]{cftool.png}
    \end{center}
    \caption{Fitting the data with cftool~\cite{luarn}}
    \label{cftool}
In the next step prediction for year 2020 was made. For prediction we use different mechanism for each model.
The first linear model Matlab's $predint$ function used to get prediction results from previous fitting function.
The second one used other approach.
Trained and prepared polynomila model was compared with real data to see the aberrance. Plotting function was used to see the results.
At a first glance we can se that the results from the models are different. The first linear model with trigonomic function
is not really god in real income results but very interesting result for trend prediction was produced.
In contrast with polynomial model which much better in absolut income results, but worst in trend prediction.
\end{figure}\\
\begin{figure}[h!]
    \begin{center}
        \includegraphics[width=80mm]{predict.png}
    \end{center}
    \caption{Predict income data in live script~\cite{luarn}}
    \label{predict}
\end{figure}\\
\subsection{Customer behavior model} \label{sec:cbm}
Our prediction script is finally written in Matlab Live script with predefined constants based on Megaplay s.r.o data,
but for modeling and dynamic prediction are used Live scripts Controls so in the script we are able to easily set all constant to model for simulate different situations.
Let see the function of our prepared model.
In the first step model get predefined data from inicial part (See more in sectin~\ref{subsec:preprocessing}).
This precalculated data entered calculation loop.
Next loop read the value of predefined number of customers in each period and simulated the virtual customer bavior in order process in the next steps:
\begin{enumerate}
    \item get probability from Vendor submodel
    \item get probability from Psychology submodel
    \item get probability from Loyalty submodel
    \item combine submodels data in Hidden Markov model
    \item read invisible states by Viterbi algoritm
    \item check result matrix produced by viterbi and save the number of success orders
    \item calculate prediction income
\end{enumerate}\\
Whole prediction is made 10 times and the final result is arithmetic mean from this ten times run to random deviation be removed.
We can see model situation in~\ref{apendixc} and that results is described in the final summary~\ref{summary} as results from previous models from section~\ref{sec:baseline}.
\subsection{Using random values} \label{subsec:rand}
In our model random generated variables was used to simulate situations from real store where the user can compare with other store in the different situation.
This other stores and situation should be better or worst to actual store.
To improve results should be better that real competitors' data from each store was used, but is not simple to get them.
This random variables are used as a simplification of that situation.
\subsection{Compare results} \label{subsec:matlab}
Lets us defined the method to compare our models results.
Absolute number of income value prediction is not to represent and important for the store owners, so we were calculated
the aberration for each month prediction and then easily calculate quarterly and yearly results.
\begin{equation} \label{eq:42}
A = ((R - P)/P) * 100,
\end{equation}\\
for predict a situation where real income is higher than prediction or for oppossite situation:
\begin{equation} \label{eq:43}
A = ((P - R)/R) * 100,
\end{equation}\\
where $A$ is aberration in percent for real income $R$ and predicted income $P$.

% !TEX root = ../thesis.tex
\chapter{Evaluation} \label{evaluation}
Whole prediction is made 10 times. The final result is arithmetic mean from this prediction which is ten times run to random deviation until is removed.
Our prediction script is finally written in Matlab Live script with predefined constants based on Megaplay s.r.o data,
    but for modeling and dynamic prediction are used Live scripts Controls so in the script we are able to easily set all constant to model for simulate different situations.
    Let see the function of our prepared model.This precalculated data entered calculation loop.
Next loop read the value of predefined number of customers in each period and simulated the virtual customer bavior in order process in the next steps:
We can see model situation in~\ref{apendixc} and that results is described in the final summary~\ref{summary} as results from previous models from section~\ref{sec:baseline}.
Then Matlab's function $predint$ was used to return income prediction results for year 2020 used fitted model in section~\ref{sec:baseline}.
to calculate aberration for each month from plotted results as it seen on figure~\ref{predict}.
\begin{figure}[h!]
    \begin{center}
        \includegraphics[width=80mm]{predict.png}
    \end{center}
    \caption{Predict income data in live script~\cite{luarn}}
    \label{predict}
\end{figure}\\
Fitted and prepared models from section~\ref{sec:baseline} was compared with real data
Experiment was simulated in Matlab, where internal predefined function was used.
See more in section~\ref{subsec:libraries}.
Let use Matlab Curve fitting tool to solve our linear and polynomial models as you can see on figure 2.4, this predefined tool calculate coefficient for and then the prediction in Matlab Live script was done as we see on figure ?? These two models was fitted on income data from years 2018 and 2019 and used in Experi- ment section 2.2.
\begin{figure}[h!]
    \begin{center}
        \includegraphics[width=90mm]{cftool.png}
    \end{center}
    \caption{Fitting the data with cftool~\cite{luarn}}
    \label{cftool}
\end{figure}\\



\section{Used software, libraries and predefined functions} \label{subsec:libraries}
\textbf{Matlab 2020a}\\
MATLAB (matrix laboratory) is a fourth-generation high-level programming language and interactive environment for numerical
computation, visualization and programming developed by MathWorks.\\
\\
\textbf{Matlab LiveScript}~\cite{livescript}\\
MATLAB live scripts and live functions are interactive documents that combine MATLAB code with formatted text, equations,
and images in a single environment called the Live Editor.
In addition, live scripts store and display output alongside the code that creates it.\\
Use live scripts and functions to:\\
\begin{itemize}
    \item Visually explore and analyze problems
    \item Share richly formatted, executable narratives
    \item Create interactive lectures for teaching
\end{itemize}\\
\\
\textbf{Matlab Curve Fitting Tool}\\
The Curve Fitting Toolbox provides a collection of GUIs and M-files.
With the toolbox we are able to data preprocessing such as sectioning and smoothing, parametric and nonparametric data fitting,
and also fit statistics which assist us in determining the goodness of fit,analysis capabilities such as extrapolation, differentiation, and integration.
A graphical environment allows you to explore and analyze data sets and fits visually and numerically.\\
\textbf{hhmgenerate}~\cite{hhmgenerate}\\
The function $hmmgenerate$ begins with the model in state 1 at step 0, prior to the first emission.
The model then makes a transition to state $i_1$, with probability $T_{1i_1}$, and generates an emission $a_k_1$ with probability $E_{i_1k_11}$.
$hmmgenerate$ returns $i_1$ as the first entry of states, and $a_k_1$ as the first entry of seq.
$[seq,states] = hmmgenerate(len,TRANS,EMIS)$ takes a known Markov model, specified by transition probability matrix $TRANS$ and emission probability matrix $EMIS$,
and uses it to generate:\\
\begin{itemize}
    \item A random sequence seq of emission symbols
    \item A random sequence states of states
\end{itemize}
The length of both $seq$ and $states$ is $len$.
$TRANS(i,j)$ is the probability of transition from state $i$ to state $j$.
$EMIS(k,l)$ is the probability that symbol $l$ is emitted from state $k$.\\
\\
\textbf{hhmviterbi}~\cite{hhmviterbi}\\
The function $hmmviterbi$ begins with the model in state 1 at step 0, prior to the first emission.
hmmviterbi computes the most likely path based on the fact that the model begins in state 1.
$STATES = hmmviterbi(seq,TRANS,EMIS)$ given a $sequence, seq$, calculates the most likely path through the hidden Markov model
specified by transition probability matrix, $TRANS$, and emission probability matrix $EMIS$. $TRANS(i,j)$ is the probability of transition from state $i$ to state $j$.
$EMIS(i,k)$ is the probability that symbol $k$ is emitted from state $i$.\\
\\
\textbf{shopycrm.com}\\
Online CRM application focused on e-commerce stores which provides all store workflows and get precalculated data which we will use for our models to simplify the prediction calculation.
\newpage
\section{Fitting models}
On the figure~\ref{plot} is graphical presentation of data and our linear and polynomial models.
It turned out that the first linear model (blue line), is not directly corresponding the data, but it is copying the trend of the income,
what we should see on figure~\ref{prediction} with predicted data to 2020.
\begin{figure}[h!]
    \begin{center}
        \includegraphics[width=140mm]{plot.png}
    \end{center}
    \caption{Fitting the model on real income from 2018 and 2018}
    \label{plot}
\end{figure}\\
\newpage
\section{Results} \label{compareresults}
\begin{table}[h!]
    \begin{center}
        \begin{tabular}{ | l | c | c | c |}
            \hline
            & \textbf{}$ \textbf{linear model} & \textbf{polynomial model} & \textbf{Customer behavior HMM}\\
            \hline
            \textbf{SSE} & 5.165.10^{12} & 2.534.10^{12} & 1.717.10^{11} \\
            \textbf{R-square} & 0.2162 & 0.6154 & 0.95 \\
            \textbf{RMSE} & 4.959.10^5 & 3.652.10^5 & 1.196.10^5\\
            \hline
        \end{tabular}
    \end{center}
    \caption{Compare results table for models.}
    \label{compare}
\end{table}\\
\begin{table}[h!]
    \begin{center}
        \begin{tabular}{ | l | c | c | c | c | c | c | c |}
            \hline
            {\textbf{Month}} & \textbf{Real income} & \textbf{1st model} & \textbf{(\%)}  & \textbf{2nd model} & \textbf{(\%)} & \textbf{3rd model} & \textbf{(\%)}\\
            \hline
            1/2021 & 2 510 086 & 2 372 333 & 6 & 2 031 000 & 24 & 2 679 419 & 7\\
            2/2021 & 1 293 778 & 2 256 560 & 74 & 1 696 000 & 31 & 1 367 816 & 6\\
            3/2021 & 1 022 762 & 2 340 092 & 129 & 1 409 000 & 38 & 1 092 252 & 7\\
            4/2021 & 1 017 408 & 2 671 671 & 163 & 1 358 000 & 38 & 1 151 301 & 13\\
            5/2021 & 1 320 608 & 3 086 545 & 134 & 1 403 000 & 6 & 1 369 985 & 4\\
            6/2021 & 1 590 878 & 3 359 684 & 111 & 1 511 000 & 5 & 1 632 372 & 3\\
            7/2021 & 1 468 808 & 3 413 621 & 132 & 1 656 000 & 13 & 1 575 324 & 7\\
            8/2021 & 1 762 959 & 3 390 580 & 92 & 1 847 000 & 5 & 1 959 480 & 11\\
            9/2021 & 1 782 582 & 3 522 629 & 98 & 1 987 000 & 11 & 1 950 306 & 9\\
            10/2021 & 1 605 396 & 3 919 220 & 144 & 2 100 000 & 31 & 1 765 984 & 10\\
            11/2021 & 2 559 070 & 4 467 471 & 75 & 2 187 000 & 17 & 2 611 529 & 2\\
            12/2021 & 2 544 271 & 4 936 855 & 94 & 2 197 000 & 16 & 2 610 694 & 3\\
            \hline
        \end{tabular}
    \end{center}
    \caption{Compare results}
    \label{Compare results}
\end{table}\\
In the graphical view we saw that all our models copy the income, but some of them are easier to calculate, this differences you can see described in summary~\ref{summary}.
\begin{figure}[h!]
    \begin{center}
        \includegraphics[width=100mm]{prediction.png}
    \end{center}
    \caption{Prediction for year 2020}
    \label{prediction}
\end{figure}\\
\newpage
\begin{table}[h!]
    \begin{center}
        \begin{tabular}{ | l | c | c | c |}
            \hline
            & \textbf{1st model} & \textbf{2nd model} & \textbf{3rd model}\\
            \hline
            1Q (\%) & 69,66 & 30,81 & 6,42\\
            2Q (\%) & 135,83 & 15,00 & 6,50\\
            3Q (\%) & 107,41 & 9,64 & 9,25\\
            4Q (\%) & 104,25 & 21,21 & 4,89\\
            \hline
        \end{tabular}
    \end{center}
    \caption{Quarterly results}
    \label{qResults}
\end{table}
\newpage

% !TEX root = ../thesis.tex

\chapter{Summary} \label{summary}
In our work we focused on creating a mathematical models to predict customer behavior based on previous store data and open data available in online tools.
As a baseline we used linear and polynomial regression which is generally used to predict this kind of data.
At the first see it's look line that the better reference is polynomial model prediction as you can see in the table with results~\ref{compareresults}, but the trend prediction from the simpliest
linear regression is very usefully for some approaches.
In the case that we have only this two models as a primary goal, it will be good idea to test their cooperation and use one to correct another.
It should have generates very interesting results with trends prediction and income with small aberration too.
Polynomial prediction has in some months worst result than we expected, this was caused by pandemic situation in the year 2020.
This situation was absolutly unpredictable.
However our set goal was to create a better mechanism to predict this important data for each business based on e-commerce solutions.
To pass our goal we used mechanisms for behavior prediction and mathematical approaches which are usually used in other industries like an autonomous driving,
telecommunications or idea which looking for correlation between introduce a new products on market with global consumer behavior.
Mechanisms and approaches we have learned during the work on this thesis give us the ability to create our own vendor, psychology and loyalty models which are the
base of the final prediction model and help us to finish the goal which we set at the beginning of the work.
Our main goal was passed by successfully prediction with aberration about 6,6\% for the year 2020.
The very good results from our model against the other one is paid by more time to prepare data to prediction model.
If the polynomial model should used more data to be trained on, it returns better results, but this was not a goal of our work.
Our goal was fulfilled in cooperation with the Megaplay s.r.o to show us the deep of his industry to set constants for our models with a minimal deflection.
Therefore the next steps should be to clearly define mechanisms to set independent industry constants only on ordinarily accessible data without consultation with business owners.
Then it will be good to minimize random generate number with a matrix create with real data of store.
Estabilished competitors matrix with real data to compare real store and vendor power and customer satisfaction.
Unfortunately this open a lots of new problems which have to be solved like a vendor strength comparation across industries.
The situation in the industry was change in an unexpected way and modeling that situation was beyond this thesis, but the results of the thesis is enough to show that standard
simple statistics approaches are not enough for prediction customer behavior in a 21st century, so we have to look on the sophisticated approaches like a mathematical modeling.
Evolution of computer technology give the power to get more complicated calculation to more users and should be used to get the more relevant data for entrepreneurs to improve their businesses.
Our prediction is actually combined ourselves models inside statistical solution like Hidden Markov model.
In the future improves it will be the right way to leave all statics mechanism and as a combination approach use differential equation which should be able to serve predition vector to get results lilke a HMM.
However when that equation will be found it open the new way of store data prediction from predefined states to dynamically updated states and show the results in realtime to reflect actual visitors and income on each store.
From different simulation using our script and update constants we get presumed results and for us not surprising results.
The number of predicted orders is dependent on numbers of store visitors.
Customer satisfaction with the combination with price index is most important value for the prediction and it's corespondent with real behavior, but this is not generally true.
It can be aplies to our Central Europe region.
However in the future improves will be good to have data from different countries or better continets to check our model in more global way.
We should expect that the weights ratio will be different in other markets, depends on specific customer psyhology behavior in different countries.


% good linebraking of bibtex url
\setcounter{biburllcpenalty}{7000}
\setcounter{biburlucpenalty}{8000}

%% The bibliography
\printbibliography[heading=bibintoc]

\label{theend} % the last page of the thesis

% List of acronyms
\printglossary[type=\acronymtype,title={\acrlistname}]

% Glossaries
%\printglossary

%% Appendix
% !TEX root = ../thesis.tex

\chapter*{List of appendix}
\addcontentsline{toc}{chapter}{List of appendix}

\begin{description}
	\item[Appendix A] Reference earn data from Megaplay s.r.o
	\item[Appendix B] Flowcharts
\end{description}

\appendix
\renewcommand\chaptername{Appendix}
% !TEX root = ../thesis.tex '

\chapter{Reference data} \label{reference_data}

As reference we will use Time series analysis especially moving average and linear regression from Megaplay s.r.o earn data. Data comes from Megaplay s.r.o online stores from years 2018 and 2019.
We will compare  our prediction model with these Time series analysis results.
\section{Megaplay s.r.o online stores earn from 2018 and 2019 per months}
\begin{table}[h!]
    \begin{center}
        \begin{tabular}{ | l | c | l | c |}
            \hline
            \textbf{Months} & \textbf{Earn} & \textbf{Months} & \textbf{Earn}\\
            \hline
            1/2018 & 1 961 088 Kč & 1/2019 & 2 012 000 Kč \\
            \hline
            2/2018 & 1 531 995 Kč & 2/2019 & 1 555 929 Kč\\
            \hline
            3/2018 & 1 689 860 Kč & 3/2019 & 1 837 794 Kč \\
            \hline
            4/2018 & 1 588 628 Kč & 4/2019 & 1 553 459 Kč\\
            \hline
            5/2018 & 1 322 548 Kč &5/2019 & 1 746 159 Kč\\
            \hline
            6/2018 & 1 494 798 Kč & 6/2019 & 1 319 472 Kč\\
            \hline
            7/2018 & 1 229 718 Kč & 7/2019 & 1 327 479 Kč\\
            \hline
            8/2018 & 1 404 787 Kč & 8/2019 & 1 295 468 Kč\\
            \hline
            9/2018 & 1 867 475 Kč & 9/2019 & 2 005 098 Kč\\
            \hline
            10/2018 & 2 350 839 Kč & 10/2019 & 1 763 850 Kč\\
            \hline
            11/2018 & 2 935 847 Kč & 11/2019 & 2 660 196 Kč\\
            \hline
            12/2018 & 2 911 601 Kč & 12/2019 & 2 887 786 Kč\\
            \hline
        \end{tabular}
    \end{center}
    \caption{Megaplay s.r.o earns from 2018 and 2019}
    \label{income_megaplay}
\end{table}

% !TEX root = ../thesis.tex

\chapter{Flowcharts}

\section{Overview}
\begin{figure}[h!]
    \begin{center}
        \includegraphics[width=350]{flowchart_overview}
    \end{center}
    \caption{Flowchart overview}
    \label{flowchart_overview}
\end{figure}
\newpage
\section{Vendor model}
\begin{figure}[h!]
    \begin{center}
        \includegraphics[width=180]{flowchart_vendor}
    \end{center}
    \caption{Flowchart vendor model}
    \label{flowchart_vendor}
\end{figure}
\section{Psychology model}
\begin{figure}[h!]
    \begin{center}
        \includegraphics[width=180]{flowchart_psychology}
    \end{center}
    \caption{Flowchart psychology model}
    \label{flowchart_psychology}
\end{figure}
\newpage
\section{Loyalty model}
\begin{figure}[h!]
    \begin{center}
        \includegraphics[width=60]{flowchart_loyalty}
    \end{center}
    \caption{Flowchart loyalty model}
    \label{flowchart_loyalty}
\end{figure}
\section{Markov hidden model}
\begin{figure}[h!]
    \begin{center}
        \includegraphics[width=230]{flowchart_markovpng}
    \end{center}
    \caption{Flowchart markov chain}
    \label{flowchart_markov}
\end{figure}
\newpage

% !TEX root = ../thesis.tex '

\chapter{Modeling different situation} \label{apendixc}

In this section we can see modeling in a different situation.

\section{Increase visitors for 10\%}
When we increase visitors for 10\% to 2592 visitors per day, it will increase our income to 2 687 593 Kč so it's incrase income about 10.47\%.
From this simulation we can see that increasing the number of visitors have positive influence to the store strength.\\
\section{Increase store strength for 5\%}
When we increase store strength for 5\%, it will increase our income to 2 447 891 Kč so it's increase income about 0.62\%.
From this simulation we can see that increasing the store strength (when the previous strength is more than 90\%) has minimal influence for our income.\\
\section{Increase Customer satisfaction for 5\%}
When we increase store Customer satisfaction for 5\%, it will increase our income to 2 435 548 Kč so it's increase income about 0.1\%.
From this calculation we can see that this increase we have to not take into account.
This prediction confirms the idea that everything more than 90\% it has minimal influence for customer decision.


% zivotopis autora
%\curriculumvitae\protect
%Táto časť\/ je nepovinná. Autor tu môže uviesť\/ svoje biografické
%údaje, údaje o~záujmoch, účasti na~projektoch, účasti na~súťažiach,
%získané ocenenia, zahraničné pobyty na~praxi, domácu prax, publikácie
%a~pod.

\end{document}
